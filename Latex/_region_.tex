\message{ !name(main.tex)}\documentclass[12pt,a4paper,english,titlepage,oneside]{scrbook}
% ***** BEGIN OF THE LATEX PREAMBLE *****

\usepackage{softeng}

% Here we define the title of the current thesis in an appropriate variable.
\newcommand{\worktitle}{A WoT Approach to eHealth}
% Here we can define a subtitle for the work. If there is no subtitle let this variable empty.
%\newcommand{\worksubtitle}{}
% otherwise you can set any subtitle you want, event long ones by breaking the lines manually:
\newcommand{\worksubtitle}{A RESTFull Web Service described in WADL \\
							and implented in Java and SQL}
			   
% This is the name of the author
\newcommand{\theauthor}{Mathieu Jean-Daniel}

% This command defines the type of the work: Master Thesis / Bachelor Thesis
\newcommand{\worktype}{Bachelor Thesis}

% Date of the end (submission) of the work.
\newcommand{\workdateyear}{2013}
\newcommand{\workdatemonth}{February}

% English: Thesis supervisors
% French: Supervis� par
% German: Unter der Aufssicht von
\newcommand{\supervisorslabel}{Thesis supervisors}

\newcommand{\worksupervisors}{
    Prof. Dr. Jacques \textsc{Pasquier-Rocha}\\
    and\\
    Andreas \textsc{Ruppen}\\
    Software Engineering Group\\
    
    % if there are other supervisor from other organisations, add them like this
    % \vspace{0.2cm}
    % Dr. Olivier \textsc{Liechti}\\
    % Sun Microsystems Inc.\\
}

% Set the titlepage footer according to the language of you report (E nglish, F rancais or D eutsch)
\newcommand{\titlepagefooter}{\titlepagefooterE}
%\newcommand{\titlepagefooter}{\titlepagefooterD}
%\newcommand{\titlepagefooter}{\titlepagefooterF}


% Here you define the labels of figures, tables, listings...
% English: Figure, Table, Listing (default)
% German: Abbildung, Tabelle, Listing
% French: Figure, Table, Listing
%\renewcommand{\figurename}{Figure}
%\renewcommand{\tablename}{Table}
%\renewcommand{\lstlistingname}{Code extract} % default = Listing



% If you are still working on a draft version, this command prints a little slight note on the top left corner of each page.
% Comment this line when you have a final version.
\reviewtimetoday{\today}{* Draft Version *}

% Let us set the path where all the figures have to be.
\graphicspath{{./figures/}}

% The following command avoids the indentation of the first sentence of each paragraph.
\parindent=0in

% The following command sets the header and footer style of the pages.
\pagestyle{fancy}

% The web resources are references in a bibliography apart.
\newcites{web}{Referenced Web Ressources}

% It is not mandatory (but suitable) to make an index. If you do not build an index: comment the line below!
% building the idx index file. This file is then to be compiled with %
% makeindex in the command line %
\usepackage{makeidx} \makeindex

% XML LstListing language settings
\definecolor{gray}{rgb}{0.4,0.4,0.4}
\definecolor{darkblue}{rgb}{0.0,0.0,0.6}
\definecolor{cyan}{rgb}{0.0,0.6,0.6}

\lstset{
  basicstyle=\ttfamily\footnotesize,
  columns=fullflexible,
  showstringspaces=false,
  commentstyle=\color{gray}\upshape
}

\lstdefinelanguage{XML}
{
  morestring=[b]",
  morestring=[s]{>}{<},
  morecomment=[s]{<?}{?>},
  stringstyle=\color{black},
  identifierstyle=\color{darkblue},
  keywordstyle=\color{cyan},
  morekeywords={xmlns,version,type}% list your attributes here
}


% ***** END OF THE LATEX PREAMBLE *****

% ***** BEGIN OF THE DOCUMENT *****
% ----- BEGIN OF THE FRONT STUFF -----
\begin{document}

\message{ !name(main.tex) !offset(-108) }
\documentclass[12pt,a4paper,english,titlepage,oneside]{scrbook}
% ***** BEGIN OF THE LATEX PREAMBLE *****

\usepackage{softeng}

% Here we define the title of the current thesis in an appropriate variable.
\newcommand{\worktitle}{A WoT Approach to eHealth}
% Here we can define a subtitle for the work. If there is no subtitle let this variable empty.
%\newcommand{\worksubtitle}{}
% otherwise you can set any subtitle you want, event long ones by breaking the lines manually:
\newcommand{\worksubtitle}{A RESTFull Web Service described in WADL \\
							and implented in Java and SQL}
			   
% This is the name of the author
\newcommand{\theauthor}{Mathieu Jean-Daniel}

% This command defines the type of the work: Master Thesis / Bachelor Thesis
\newcommand{\worktype}{Bachelor Thesis}

% Date of the end (submission) of the work.
\newcommand{\workdateyear}{2013}
\newcommand{\workdatemonth}{February}

% English: Thesis supervisors
% French: Supervis� par
% German: Unter der Aufssicht von
\newcommand{\supervisorslabel}{Thesis supervisors}

\newcommand{\worksupervisors}{
    Prof. Dr. Jacques \textsc{Pasquier-Rocha}\\
    and\\
    Andreas \textsc{Ruppen}\\
    Software Engineering Group\\
    
    % if there are other supervisor from other organisations, add them like this
    % \vspace{0.2cm}
    % Dr. Olivier \textsc{Liechti}\\
    % Sun Microsystems Inc.\\
}

% Set the titlepage footer according to the language of you report (E nglish, F rancais or D eutsch)
\newcommand{\titlepagefooter}{\titlepagefooterE}
%\newcommand{\titlepagefooter}{\titlepagefooterD}
%\newcommand{\titlepagefooter}{\titlepagefooterF}


% Here you define the labels of figures, tables, listings...
% English: Figure, Table, Listing (default)
% German: Abbildung, Tabelle, Listing
% French: Figure, Table, Listing
%\renewcommand{\figurename}{Figure}
%\renewcommand{\tablename}{Table}
%\renewcommand{\lstlistingname}{Code extract} % default = Listing



% If you are still working on a draft version, this command prints a little slight note on the top left corner of each page.
% Comment this line when you have a final version.
\reviewtimetoday{\today}{* Draft Version *}

% Let us set the path where all the figures have to be.
\graphicspath{{./figures/}}

% The following command avoids the indentation of the first sentence of each paragraph.
\parindent=0in

% The following command sets the header and footer style of the pages.
\pagestyle{fancy}

% The web resources are references in a bibliography apart.
\newcites{web}{Referenced Web Ressources}

% It is not mandatory (but suitable) to make an index. If you do not build an index: comment the line below!
% building the idx index file. This file is then to be compiled with %
% makeindex in the command line %
\usepackage{makeidx} \makeindex

% XML LstListing language settings
\definecolor{gray}{rgb}{0.4,0.4,0.4}
\definecolor{darkblue}{rgb}{0.0,0.0,0.6}
\definecolor{cyan}{rgb}{0.0,0.6,0.6}

\lstset{
  basicstyle=\ttfamily\footnotesize,
  columns=fullflexible,
  showstringspaces=false,
  commentstyle=\color{gray}\upshape
}

\lstdefinelanguage{XML}
{
  morestring=[b]",
  morestring=[s]{>}{<},

\message{ !name(main.tex) !offset(4) }

\end{document}
% ***** END OF THE DOCUMENT *****
