\chapter{Introduction to Agile}
This chapter gives an overview over agile software development processes in general.\\

In the traditional software development there are many people working on the requirements of the future software. When this is done, the developers start to work on it. During this phase there is usually no feedback from either the product owner or future users. The developers work more or less on their own which usually results in a disaster. (Quelle: http://www.galorath.com/wp/software-project-failure-costs-billions-better-estimation-planning-can-help.php) According to blabla only 32 percent of the IT projects in were successful. This needs to be changed and agile methodologies are trying to do that by trying to produce something valuable instead of over-planning.

\section{The Agile Processes}
The agile software development processes work in iterations. These are repeated at least until the software is shippable. A very important aspect of the process is the fact that after each iteration there should be an increment of the project which has some value and would be shippable.


\subsection{Limits}
According to \cite{turk2002limitations} using the agile software development process brings a few drawbacks and limitations. 
\begin{enumerate}
\item Limited support for distributed development environments.
\item Limited support for subcontracting.
\item Limited support for building reusable artefacts.
\item Limited support for development involving large teams.
\item Limited support for developing large, complex software.
\item Limited support for developing safety-critical software.
\end{enumerate}
In chapter blabla we discuss if these limitations also apply to our agile start up process.

\cite{scrum_linda}